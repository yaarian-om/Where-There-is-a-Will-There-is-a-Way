\documentclass{article}
\usepackage{color}
\usepackage{xcolor}
\usepackage{graphicx}
\usepackage{graphics}                       %  For Inserting List of Figures
\usepackage{index}                          %  For Inserting Index in the document
\usepackage{url,hyperref}                   %  For Inserting Index in the document
\usepackage[xindy]{imakeidx}                %  For Inserting Index in Alphabetical order
%\usepackage{float}                         %  For Preventing Table Repositioning
\usepackage{float}                          %  For Preventing Table Repositioning
% \usepackage{geometry}
\usepackage[super]{nth}                     % This is needed to use SuperScripted 1st 2nd 3rd 4th ...
% \usepackage[a4paper,showframe]{geometry}    % Change while production %%%% This makes the outline frame of the book


\begin{document}

\begin{titlepage}
    \begin{center}

        \vspace*{\fill}
        \Huge
        \textbf{COMPUTER GRAPHICS ASSIGNMENT} \\
        \vspace*{\fill}

        \vspace*{\fill}
        \raggedright
        \Large
        \underline{SUBMITTED BY:} \\
        \vspace*{0.2cm}
        Name : SUDIPTA KUMAR DAS \\
        ID   : 20-43658-2 \\
        SEC  : J \\
        DEPT. OF COMPUTER SCIENCE \\
        \vspace*{\fill}

        \vspace*{\fill}
        \raggedleft
        \Large
        \underline{SUBMITTED TO:} \\
        \vspace*{0.2cm}
        DR. MD. TAIMUR AHAD \\
        FACULTY
        \vspace*{\fill}


    \end{center}
\end{titlepage}

\tableofcontents
% \newpage
\listoffigures
% \newpage
\listoftables

\newpage

\section{Phong and Gouraud shading: The differences in  approach and method.}
\subsection{Gouraud shading}
A per-vertex color calculation is used in gouaud shading, also known as smooth
shading. This implies that the fragment shader will get the color that the vertex
shader assigns to each vertex as an out variable. This color is interpolated across
the fragments to provide the smooth shading since it is supplied to the fragment
shader as an invariant variable.
\begin{figure}[htbp]
    \begin{center}
        \includegraphics*[width=8cm]{gourad.jpg}
        \caption{Gouraud Shading}
    \end{center}
\end{figure}

\subsection{Phong shading}
A per-fragment color calculation is Phong shading. The fragment shader receives
the normal and position information from the vertex shader as out variables.
The color is subsequently calculated by the fragment shader using interpolation
between these variables.

\begin{figure}[htbp]
    \begin{center}
        \includegraphics*[width=8cm]{phong.jpg}
        \caption{Phong Shading}
    \end{center}
\end{figure}

\newpage

\subsection{The differences in Phong and Gouraud shading}
\begin{table}[htbp]
    \begin{tabular}{|l|l|}
        \hline
        \multicolumn{1}{|c|}{Gouraud Shading}                                                                                                                                                      & \multicolumn{1}{c|}{Phong Shading}                                                                                                                                                                         \\ \hline
        \begin{tabular}[c]{@{}l@{}}The Gourard shading method falls \\ somewhere in the middle of the two: \\ like Phong shading, each polygon has \\ a regular vector at each vertex\end{tabular} & \begin{tabular}[c]{@{}l@{}}Each vertex of a drawn polygon has a typical \\ vector, which is added to the surface to execute \\ shading and determine the color for each point \\ of interest.\end{tabular} \\ \hline
        This kind of shading is not expensive                                                                                                                                                      & \begin{tabular}[c]{@{}l@{}}This kind of shading is more expensive than \\ Gouraud Shading\end{tabular}                                                                                                     \\ \hline
        \begin{tabular}[c]{@{}l@{}}Takes a moderate amount of time and\\ processing.\end{tabular}                                                                                                  & \begin{tabular}[c]{@{}l@{}}It is slower and requires complex processing. \\ Its products are high caliber.\end{tabular}                                                                                    \\ \hline
        Gleaming surfaces                                                                                                                                                                          & Surfaces with a polished finish.                                                                                                                                                                           \\ \hline
        Each vertex uses the lighting equation                                                                                                                                                     & Each pixel makes use of the lighting equation.                                                                                                                                                             \\ \hline
        \begin{tabular}[c]{@{}l@{}}Interpolates and computes illumination \\ at boundary verticies\end{tabular}                                                                                    & \begin{tabular}[c]{@{}l@{}}Every point on the surface of the polygon is \\ illuminated.\end{tabular}                                                                                                       \\ \hline
        \begin{tabular}[c]{@{}l@{}}The methodology was initially described \\ in 1971 by Gouraud\end{tabular}                                                                                      & In 1973, Phong Shading published the method.                                                                                                                                                               \\ \hline
        \begin{tabular}[c]{@{}l@{}}Henri Gouraud is the namesake of the \\ Gouraud shading style\end{tabular}                                                                                      & \begin{tabular}[c]{@{}l@{}}After Bui Tuong Phong, the Phong Shading \\ the model was created.\end{tabular}                                                                                                 \\ \hline
    \end{tabular}
    \centering
    \caption{Gouraud Shading VS Phong Shading}
\end{table}

\subsection{Example of Gouraud and Phong shading}
\begin{figure}[htbp]
    \begin{center}
        \includegraphics*[width=6.6cm]{SHADING.png}
        \caption{Gouraud Shading VS Phong Shading}
    \end{center}
\end{figure}


\end{document}